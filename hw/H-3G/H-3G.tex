\documentclass{article}
\usepackage{../fasy-hw}
\usepackage{ wasysym }

%% UPDATE these variables:
\renewcommand{\hwnum}{3}
\title{Advanced Algorithms, Homework \hwnum}
\author{TODO-Group Member Names Here}
\collab{n/a}
\date{due: 17 September 2020}

\begin{document}

\maketitle

This homework assignment is due on 3 September 2020, and should be
submitted as a single PDF file to to Gradescope.

General homework expectations:
\begin{itemize}
    \item Homework should be typeset using LaTex.
    \item Answers should be in complete sentences and proofread.
    \item This homework can be submitted as a group.
\end{itemize}

\nextprob
\collab{TODO}

Work in a group of size $\geq 2$.  Explain your strategy for working in a group.

\paragraph{Answer}

% ============================================

Ryan and I will employ the standard approach, which will be to work together
starting from the beginning of the assignment, each tackling problems individually and
checking eachother's work on the problems that we didn't initially work on. Then for problems
that we both don't know how to do right away, we often brainstorm together. Normally it works
quite well for us to just chip away at an assignment together, individually until we are unclear on
certain problems.

% ============================================

\nextprob
\collab{TODO}

Your group should make at least five contributions to the Piazza board.  A
contribution can be either asking a relevant question, responding to another
student's question, responding to an instructor's question, or choosing a
question from Chapter 1 and attempting to solve it, then  describing where you
get stuck in answering it.

\paragraph{Answer}

% ============================================

Our groups contributions are:
\begin{enumerate}
	\item (The post was originally from Dr. Fasy on 9/2 regarding a question from the activity in class) TODO:
        copy the post here.
    \item (TODO: state the problem number, name of poster, and date/time). TODO:
        copy the post here.
    \item (TODO: state the problem number, name of poster, and date/time). TODO:
        copy the post here.
    \item (TODO: state the problem number, name of poster, and date/time). TODO:
        copy the post here.
    \item (TODO: state the problem number, name of poster, and date/time). TODO:
        copy the post here.
\end{enumerate}

% ============================================

\nextprob
\collab{TODO}

Give the algorithm for binary search, using a for loop and no recursion.

\begin{enumerate}
    \item Describe the problem in your own words, including
        describing what the input and output is.
    \item Describe, in paragraph form, the algorithm you propose.
    \item Provide this algorithm in the algorithm environment.
    \item Use a decrementing function to prove that the loop terminates.
    \item What is the loop invariant? Provide the proof.
\end{enumerate}

\paragraph{Answer}

% ============================================

\begin{enumerate}
	\item The problem takes as input a sorted array and a proposed value in that array, and returns
		as output the boolean value corresponding to whether or not the value exists in the array.
		We assume that the array is sorted from smallest to largest, with the value at index 1
		being the smallest, and the last index containing the largest item.
	\item The algorithm we propose uses a for loop which may iterate as many times as there are items in
		the input array. On the first iteration, we compare the input value to the value at the middle
		index. If the input equals the midpoint value, we break out of the for loop and return true.
		If the input is larger than the value at the middle index, we only consider the largest
		half of the array for the next iteration. If the input is smaller than the midpoint, we only
		consider the smallest half of the array for the next iteration. Then on the next iteration,
		we compute the midpoint of the new subarray and continue the process. If the input is eventually
		found, we break out of the for loop and return true. Otherwise, once we eventually reach
		a list of length 1, we either return true if this last item is the input, and false otherwise.
		For either case, we break out of the for loop. Midpoints are calculated on odd length arrays
		by rounding down (integer division).
        \item
                \begin{algorithm}\caption{\textsc{BinarySearch}($A$, $\omega$)}\label{alg:bs}
                        {\bf Input:} an array $A$ sorted from smallest at index 0 to largest at the last index,
                        proposed value in array $\omega$\\
                        {\bf Output:} Return true if $\omega \in A$. Otherwise, return false.
                \begin{algorithmic}[1]

			\State $current_A \gets A$
                        \State $size_A \gets |current_A|$
			\State $n \gets |A|$

			\While{$size_A \geq 1$}
				\If{$size_A == 1$}
					\If{$\omega == current_A[1]$}
						\State \Return $true$
					\Else
						\State \Return $false$
					\EndIf
				\EndIf
				\State $midpoint \gets size_A/2$
				\If{$\omega == current_A[midpoint]$} 
                              		\State \Return $true$
				\EndIf
				\If{$\omega > current_A[midpoint$}
					\State $current_A \gets current_A[midpoint,...,size_A]$
					\State $size_A \gets |current_A|$
				\ElsIf{$\omega < current_A[midpoint]$}
					\State $current_A \gets current_A[1,...,midpoint]$
					\State $size_A \gets |current_A|$
				\EndIf
                        \EndWhile

                \end{algorithmic}
                \end{algorithm}
	\item 
		\textbf{Claim:} The loop in \textsc{BinarySearch} terminates.
		\begin{proof}
			Let $\mathcal{I}$ denote the set of all indices within the input array $A$, but instead of
			ranging from 1 to $n$ as in the algorithm above, let $\mathcal{I}$ range from 0 to $n-1$. Notice that
			$\mathcal{I} \subset \mathbb{Z}$. Define $f:\mathbb{Z}\to\mathbb{Z}$ where for any $i \in \mathcal{I}$,
			$f(i) = (n-1)-i$. Thus switching the ordering of $\mathcal{I}$. In our for loop, evaluate $f(i)$ for any
			possible value $i \in \mathcal{I}$. At first, when $i=0$, $f(i) = n - 1$. As $i$ increases, $f(i)$ 
			decreases. No other loops exist within the main for loop in the algorithm, so eventually $i = n-1$ is reached.
			Then, $f(n-1) = 0$. Because our decrementing function reaches zero as the final possible index in the
			for loop is reached, we've shown that the for loop terminates.
		\end{proof}
	\item
		\textbf{Claim:} At the start of each iteration of the while loop in lines 5-24 of \textsc{BinarySearch}, if a search input value
		$s$ is inside of the input array $A$, it is inside the subset of $A$ currently being considered as $current_A$.
		\begin{proof}
			To prove the loop invariant, we need to show that this invariant is true before the first iteration of the loop, that
			the invariant is maintained with every iteration of the loop, and that a correctness property comes out upon the
			termination of the loop.

			First, consider the preconditions of the loop. Before the first iteration of the loop, $A$ is sorted, and $current_A$ 
			is just all of $A$. We know that the loop invariant is true before the first iteration of the loop, since indeed, if
			$s \in A$, $s \in current_A$ as $A=current_A$ for the first iteration.

			Next, consider the maintenance of the loop invariant. Observe that $current_A$ is reassigned to half of the last iteration
			of $curent_A$ on any given iteration, and this is done corresponding to the value of $s$ with respect to the value of the
			midpoint of the last iteration's $current_A$. Specifically, if $s$ is greater than the midpoint of $current_A$, we assign
			$current_A$ for the next iteration to be the larger half of $current_A$. If $s$ is smaller than the midpoint of $current_A$,
			we assign $current_A$ for the next iteration to be the smaller half of $current_A$. In either case, if $s \in A$, it is also
			necessarily in $current_A$ throughout the entire mainenance step, since for either partition we are guaranteed to have a value
			bounding $s$ on one extremity, and consider the entire remaining portion of the current array up to this bound.

			Finally, consider the termination of the loop. At termination, the loop has returned either
			a value of true or false. The loop guard continually keeps checking smaller and smaller values as $size_A$ is split in half
			on every iteration, and eventually a value of $size_A=1$ must be reached, since any value greater than 1 divided by two must 
			be $\geq 1$. In this case which must be reached, we must either return true or false. Once the loop terminates, we've either 
			then found the value $s$ as a midpoint in the partition, or we've
			continued to make $current_A$ smaller and smaller until it only contains one element. If this element is $s$, we return true, and
			otherwise we've unearthed that $s \not \in A$, and we return false. Thus, by termination, the loop invariant remains true and
			reveals to us the key property of whether or not $s \in A$.
		\end{proof}


\end{enumerate}


% ============================================

\nextprob
\collab{TODO}

Chapter 2, Problem 1b (Generalized \textsc{SubsetSum}).
\begin{enumerate}
    \item Describe the problem in your own words, including
        describing what the input and output is.
    \item Describe, in paragraph form, the algorithm you propose.
    \item Provide this algorithm in the algorithm environment.
    \item What is the runtime of your algorithm?
    \item Prove partial correctness (that if your algorithm terminates, it is
        correct).
\end{enumerate}



\paragraph{Answer}

% ============================================

\begin{enumerate}
	\item
		We are given an input of two arrays $X$ and $W$, where $X$ consists of a set of positive integers,
		and $W$ consists of a set of positive integers which correspond as weights to $X$, and a positive
		integer $T$. With this input, we want to compute the maximum weighted subset of $X$ whose elements 
		sum to the given $T$. This maximal subset of $X$ is provided as output if there exists one or more
		subsets of $X$ which sum to $T$, and otherwise $-\infty$ is provided as output when no subset of
		$X$ sums to $T$.
	\item
		To accomplish this somewhat tricky twist on \textsc{SubsetSum}, we base our algorithm on the original
		\textsc{SubsetSum} algorithm, only we keep track of lists which sum to an integer rather than boolean
		values which indicate whether or not a subset sum is possible. We call this algorithm 
		\textsc{SpecificSubsetSum}, which is a subalgorithm of the overall algorithm which we call
		\textsc{WeightedSubsetSum}. In \textsc{SpecificSubsetSum}, we first go through the array as it is
		provided, and take as input $X, i, W, X', and T$ where $X, W,$ and $T$ are defined as above and $i$
		corresponds to the index of the next value we're considering, where $i$ ranges from 1 to $|X|$
		and is initialized to $|X|$,
		and $X'$ corresponds to the actual subset that's been so far considered since the first recursion 
		which is initialized as $\emptyset$.
		If $T=0$, we return $X'$  as a base case. Otherwise, if $T<0$ or $i = 0$ we return $-\infty$ as a
		second base case, because we know we can't find a subset sum using only positive values to get to
		a negative number, and if we reach $i=0$ we know that we are unable to check any more values in $X$
		since $X$ has indices ranging from 1 to $|X|$, so we must return that a subset sum is not possible.
		Otherwise, outside of the base cases, we recursively call the algorithm both with and without the
		current index. Then we just return the maximally weighted subset returned from the recursive calls,
		so long as we find a subset which is not $-\infty$. Note that we assume linkages in memory between
		weights and corresponding values in $X$ and $W$. We denote the set of weights for any given set $X$
		as $W_{X}$.

		To assure that we're finding every possible subset, we compute this modified algorithm 
		on many more sets than just the original $X$. 
		Namely, we compute the set of all possible permutations of $X$. Letting $|X|=n$, we denote 
		the $n!$ total permutations of $X$ as $X^*_1, ... , X^*_{n!}$ and the corresponding weights 
		of permutations comprising the sets $W^*_1,...,W^*_{n!}$. Permutations are computed by recording
		every possible combination of the indices of $X$.
		Then we compute
                \textsc{SpecificSubsetSum} on this gigantic set of all $X^*$ storing every outputted subset which is
		not $-\infty$ in a new set, $X_{candidate}$ and its corresponding summed weight in $W_{candidate}$. In 
		doing so, we've allowed the evaluation of
		every possible ordering of $X$, meaning that we will find every possible subset which sums to $T$.
                Then, for every candidate in $X_{candidate}$, we compute the maximum weight, and return the candidate
                subset corresponding to this maximal weight.

	\item
		\begin{algorithm}\caption{\textsc{SpecificSubsetSum}($X$, $X'=\emptyset$ $W$, $i=|X|$, $T$)}\label{alg:sss}
                        {\bf Input:} A set $X$ with corresponding weights $W$, a subset $X' \subset X$ initialized to $\emptyset$ unless
			otherwise specified, an index $i$ initialized to $|X|$ unless otherwise specified, and a positive integer $T$\\
                        {\bf Output:} Return a subset of $X$ which sums to $T$ if any exist. If not, return $-\infty$.
                \begin{algorithmic}[1]

			\If{$T=0$}
				\State \Return $X'$
			\ElsIf{$T<0$ or $i=0$}
				\State \Return $-\infty$
			\Else
				\State $with \gets \textsc{SpecificSubsetSum}(X, X' + X[i], W' + W[i], i-1, T-X[i])$
				\State $wout \gets \textsc{SpecificSubsetSum}(X, X', W', i-1, T)$
				\If{$wout \not= -\infty \and with \not= -\infty$}
					\State $sumWout = W_{wout}[1] + ... + W_{wout}[|W|]$
					\State $sumWith = W_{with}[1] + ... + W_{with}[|W|]$
					\State $max \gets max(sumWout, sumWith)$
					\If{$max == sumWout$}
						\State \Return $wout$
					\Else
						\State \Return $with$
					\EndIf
				\ElsIf{$wout \not=-\infty$}
					\State \Return $wout$
				\ElsIf{$with \not=-\infty$}
					\State \Return $with$
				\Else
					\State \Return $-\infty$
				\EndIf
			\EndIf

                \end{algorithmic}
                \end{algorithm}

		\begin{algorithm}\caption{\textsc{WeightedSubsetSum}($X$, $W$, $T$)}\label{alg:sss}
                        {\bf Input:} A set $X$ with corresponding weights $W$, and a positive integer $T$\\
                        {\bf Output:} Return the maximally weighted subset of $X$ which sums to $T$ if any exist. 
			If not, return $-\infty$.
                \begin{algorithmic}[1]
			\State $X^*_1, ... , X^*_{|X|!} \gets$ All possible permutations of $X$
			\State $W^*_1, ..., W^*_{|X|!} \gets$ the corresponding weights to all permutations of $X$
			\State $i \gets 1$
			\State $n \gets |X|$
			\State $max \gets \{-\infty\}$
			\State $maxVal \gets -\infty$
			\For{$i < n!$}
				\State $current \gets$ \textsc{SpecificSubsetSum}($X^*_i, \emptyset, W^*_i, |X^*_i|, T$)
				\State $currentVal \gets W_{current}[1] + ... + W_{current}[|W_{current}|]$
				\If{$currentVal > maxVal$}
					\State $maxVal \gets currentVal$
					\State $max \gets current$
				\EndIf
			\EndFor
			\State \Return $max$

		\end{algorithmic}
		\end{algorithm}

	\item
		\textsc{SpecificSubsetSum} has the same runtime as \textsc{SubsetSum} as its presented in the textbook,
		since the algorithm takes identical form, only it returns the sets which are guaranteed to sum to $T$
		rather than a boolean value. This runtime is shown to be $O(2^n)$ on page 78 of Jeff Erickson's book.
		Then, \textsc{WeightedSubsetSum} conducts $n!$ runs of \textsc{SubsetSum}, as it runs on every possible
		permutation of $X$. Making the runtime of \textsc{WeightedSubsetSum} $=O(2^nn!)$, which isn't especially
		speedy. We would expect a high runtime however, since subset sum is already an NP-complete problem,
		and we've only made it more complicated by adding in weights.
	\item
		\textbf{Claim:} If our algorithm terminates, it is correct.
		\begin{proof}
			To prove partial correctness, we need to show prove the loop invariant of our algorithm. We can
			guarantee correctness of \textsc{SpecificSubsetSum}, because it is identical algorithmically to
			\textsc{SubsetSum}, only it returns the actual subset rather than a boolean value. Then our loop
			invariant can be found in \textsc{WeightedSubsetSum}, in the loop contained in lines 7 through 14.
			Making our loop invariant that $max$ is storing the largest weighted subset of $X$ summing to $T$
			which has been passed over in the loop, eventually making $max$ necessarily the largest weighted subset of $X$
			which sums to $T$. So then we need to show that the loop invariant holds throughout initialization,
			maintenence, and termination.

			Certainly the loop invariant holds for initialization, as the value $max$ is empty, which has
			the largest weight sum of anything we've iterated through, since we haven't iterated through anything.

			For the maintenence step, we can also see that the loop invariant holds, as $max$ is updated whenever
			a new value is encountered with a weight sum larger than $max$. This means that the largest weight sum
			that has yet been encountered in the loop must be stored in $max$, otherwise it would've been updated in
			line 12, with the value of its weight sum stored for future comparisons in line 11.

			Finally, upon termination, the loop has iterated over every possible permutation of $X$, and has either
			computed a subset which sums to $T$, or has returned $-\infty$. Because the loop maintenance holds, $max$
			must have the largest weighted subset summing to $T$ out of every value 
			that's been iterated on, (which in this case is every value), or a value 
			of $-\infty$. Since every permuted set we've iterated over has comprised every possible permutation of $X$,
			running \textsc{SpecificSubsetSum} must unearth every possible subset which sums to $T$. And since we can guarantee
			that the subset in $max$ must be the largest weighted subset with respect to everything we've iterated over,
			it reveals the critical property of \textsc{WeightedSubsetSum} that by the end of the loop we've identified the
			largest weighted subset which sums to $T$. If none exist, we must've only found weight sums also equaling $-\infty$,
			so this is what must be returned in the trivial case.
		\end{proof}






\end{enumerate}

% ============================================



\nextprob
\collab{TODO}

Describe two different data structures that you can use to store a graph.
Please give a complete description (i.e., a response of ``an array'' will not
suffice).

\paragraph{Answer}

% ============================================

\begin{enumerate}
	\item
		One option of a data structure for a graph is a 2-dimensional array. Each entry of the array in the first
		dimension corresponds to a vertex in the graph, and the second dimension for each item is then every vertex
		which this vertex shares an edge with.
	\item
		A second option of a data structure for a graph could be an $n$ x $n$ matrix for a graph with $n$ vertices.
		This matrix is comprised of rows and columns each corresponding to vertices, with rows corresponding to the
		number of edges shared with each vertex for any given vertex. The diagonal then, is comprised of the number
		of self edges for any vertex. These matrices are easily implementable using a 2-dimensional array as
		described above if a matrix hasn't already been implemented. The key difference however is that each row
		corresponds to the connections between that row's corresponding vertex and all of the other vertices, whereas
		a 2-D array only generally keeps track of edges between vertices and not also directly the absence of edges,
		as is done by the 0-entries in an adjacency matrix.
\end{enumerate}

% ============================================


\nextprob
\collab{TODO}

Walk through the exponential time Longest Increasing Subsequence (LIS) algorithm
on page 108 for the input: $\left[ 1, 7, 6, 11, 3, 11 \right]$.

Walk through the algorithm using the Dynamic Programming algorithm present in
Section 3.6.

\paragraph{Answer}
\section{Recursion}

\begin{enumerate}
	\item lisbigger(0,1): 1 vs 7: max return 3
	\item - lisbigger(0,2): 1 vs 6: max return 3
	\item - - lisbigger(0,3): 1 vs 11: max return 2
	\item - - - lisbigger(0, 4): 1 vs 3: max return 2
	\item - - - - lisbigger(0,5): 1 vs 11: max return 1
	\item - - - - - lisbigger(0,6): return 0
	\item - - - - - 1 + lisbigger(5, 6): return 1
	\item - - - - 1 + lisbigger(4, 5): 3 vs 11: max return 2
	\item - - - - - lisbigger(4, 6): return 0
	\item - - - - - 1 + lisbigger(5, 6): return 1
	\item - - - 1 + lisbigger(3, 4): 11 vs 3: 2nd case return 1
	\item - - - - lisbigger(3, 5): 11 vs 11: 2nd case return 0
	\item - - - - - lisbigger(3, 6): return 0
	\item - - 1 + lisbigger(2, 3): 6 vs 11: max return 2
	\item - - - lisbigger(2, 4): 6 vs 3: case 2 return 1
	\item - - - - lisbigger(2, 5): 6 vs 11: max return 1
	\item - - - - - lisbigger(2, 6) return 0
	\item - - - - - 1 + lisbigger(5, 6): return 1
	\item - - - 1 + lisbigger(3, 4): 11 vs 3 (ive computed this earlier) returns 1
	\item - 1 + lisbigger(1, 3): 7 vs 6 case 2 return 3
	\item - - lisbigger(1, 4): 7 vs 11: max return 2
	\item - - - lisbigger(1, 4): 7 vs 3: case 2 return 1
	\item - - - - lisbigger(1, 5): 7 vs 11: max return 1
	\item - - - - - lisbigger(1,6): return 0
	\item - - - - - 1 + lisbigger(5,6) return 1
	\item - - - 1 + lisbigger(3, 4): (done before) return 1

\end{enumerate}
Meaning the longest increasing substring is a length of 3
I will admit I did use the dynamic approach myself for my sanity

\section{Dynamic}

\begin{enumerate}
	\item Create base cases --- store 0 in lisbigger[m, 6] for any integer m. 
	\item Generate values for each column in our lisbigger[] table from right to left
	\item Lisbigger[0,5] = 1
	\item Lisbigger[1,5] = 1
	\item Lisbigger[2,5] = 1
	\item Lisbigger[3,5] = 0
	\item lisbigger[4,5] = 1
	\item lisbigger[0,4] = 2
	\item lisbigger[1,4] = 1
	\item lisbigger[2,4] = 1
	\item lisbigger[3,4] = 0
	\item lisbigger[0,3] = 2
	\item lisbigger[1,3] = 2
	\item lisbigger[2,3] = 1
	\item lisbigger[0,2] = 3
	\item lisbigger[1,2] = 2
	\item lisbigger[0,1] = 3
\end{enumerate}



\nextprob
\collab{TODO}

What is the closed form of the following recurrence relations?  Use Master's
theorem to justify your answers:
\begin{enumerate}
    \item $T(n) = 16 T(n/4) + \Theta(n)$
    \item $T(n) = 2 T(n/2) + n \log{n}$
    \item $T(n) = 6 T(n/3) + n^2 \log{n}$
    \item $T(n) = 4 T(n/2) + n^2$
    \item $T(n) = 9 T(n/3) + n$
\end{enumerate}
Note: we assume that $T(1)=\Theta(1)$ whenever it is not explicitly given.

\paragraph{Answer}

% ============================================

TODO: your answer goes between these lines

% ============================================

\nextprob
\collab{TODO}

\emph{The skyline problem:} You are in Camden, NJ waiting for the ferry across the river to
get into Philadelphia, and are looking at the skyline.  You take a photo, and notice that each building
has the silhouette of a rectangle.  Suppose you  represent each building $b$ as a
triple $(x_b^{(1)},x_b^{(2)},y_b)$, where the building can be seen from $x_b^{(1)}$ to $x_b^{(2)}$
horizontally and has a height of $y_b$.  Let $\mathtt{rect(b)}$ be the set of
points inside this rectangle (including the boundary).  Let $\mathtt{buildings}$
be a set of $n$ such triples representing buildings. Design an algorithm that takes $\mathtt{buildings}$ as input, and
returns the skyline, where the skyline is a sequence of~$(x,y)$ coordinates
defining $\cup_{b \in \mathtt{buildings}} \mathtt{rect}(b)$.  The output should
start with $(\min_b{x_b^{(1)}},0)$ and end with $(\max_b{x_b^{(1)}},0)$.

\begin{enumerate}
    \item Describe the problem in your own words, including
        describing what the input and output is.
    \item Describe, in paragraph form, the algorithm you propose.
    \item Provide this algorithm in the algorithm environment.
    \item What is the runtime of your algorithm? If you do not know, either give
        the tightest bounds you know, or provide a decrementing function to show
        that it does terminate.
    \item Prove partial correctness (that if your algorithm terminates, it is
        correct).
\end{enumerate}



\paragraph{Answer}

% ============================================

\begin{enumerate}
	\item We are given an x,y graph with a bunch of rectangles on it. If we union all the rectangles together, we are basically trying to find the boundary of the shape. Since all the shapes are rectangles, though, we can describe the boundary as a sequence of points that are connected via straight lines. The input would be the shape (photo) and the output would be the sequence of points (skyline). 
	\item We build up a solution by adding rectangles 1 by 1 to our silhouette. When we add a rectangle, we query if the skyline's points are interior or exterior to it. If the skylines points are all exterior to the rectangle, we know the rectangle does not contribute to the skyline. If any of the skyline's points are interior, we remove that point and expand our skyline. Additionally, we check if the rectangle has exterior points to the skyline. We add these points to the skyline.  Cool problem!
	\item 
\begin{algorithm}\caption{\textsc{Skyliner}($R$)}\label{alg:bs}
                        {\bf Input:} List of rectangles $R$,
                        {\bf Output:} Return list of points in skyline $S$ 
                \begin{algorithmic}[h]
			\State -----\bf{FUNCTION IsInterior(rectangle b, point point}

		\If{point.x $\geq x_b^1$
		and point.x $\leq$ $x_b^2$}
		
			\If{point.y $\leq$ $y_b$}
			\State \Return True
			\EndIf
			
		
		\Else
			\State \Return False
		\EndIf

		\State End Function
			\State -------


	\State -----\bf{FUNCTION AddRectangle(rectangle b, Skyline $S$}
			\For{p in $S$}
			\If{IsInterior(b,p)}
			\State ExpandSkyline(b, $S$, p)
			\EndIf
			\EndFor
			\For{q in b}
			\If{IsInterior(q,$S$)}
			\State addpoint(q,$S$)
			\EndIf
			\EndFor
	\State End Function
	\State -------

	\State -----\bf{FUNCTION ExpandSkyline(rectangle b, Skyline $S$, point p}
			\State replace p in $S$ with $(p.x, b.y)$ and $(b.x, p.y)$
\State End Function
	\State -------




		\State $S \gets \emptyset$ 




		\end{algorithmic}
                \end{algorithm}
	\item  If there are n buildings, we have 4n original points. Then we are manipulating O(n) points in our skyline O(n) times, once for each added rectangle. Then our algorithm is $O(n^2)$. I am suspicious we could prove it is $O(nlogn)$ but this still satisfies termination.
	\item Basically a proof of our loop invariant on the for loop that iterates through all the rectangles - Each loop returns the skyline for the first n rectangles. Proof - \\
		Base case: The first rectangle has all of its points in the skyline, so it is trivially true. \\
		Inductive Step - Suppose we have the skyline for the first $n$ rectangles. We want to show our for loop returns the skyline for the first $n+1$ rectangles. I will make this logical jump with contradiction. Suppose our algorithm did not return the proper skyline. Then either there is a point in our skyline that should not be, or there is a point not in our skyline that should be. \\
		Case 1 - Point in our skyline that should not be. If a point in our skyline should not be there, it is an interior point. Then, since we verify every point in our skyline is not an interior point for every rectangle we add, we cannot fail this way!\\

		Case 2 - Point not in our skyline that should be. If there is a point that is part of our skyline, it is guaranteed to be an exterior point on a rectangle. Otherwise, it would not be part of the boundary. Since we check all of these points for each rectangle, we don't fail this way either.\\
		Note: When we add points, we add them sorted by x. We can verify they are in the right order because we must draw straight lines to them. 
		
		Then indeed, if our algorithm terminates, it returns the correct skyline.

        		
\end{enumerate}
% ============================================



\end{document}


