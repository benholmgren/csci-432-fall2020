\documentclass{article}
\usepackage{../fasy-hw}
\usepackage{ wasysym}

%% UPDATE these variables:
\renewcommand{\hwnum}{2}
<<<<<<< HEAD
\title{Advanced Algorithms, Homework 1}
\author{Ben Holmgren}
=======
\title{Advanced Algorithms, Homework \hwnum}
\author{TODO-Group Member Names Here}
>>>>>>> 6ef736806df025e7a80b449def37dee559e01d9a
\collab{n/a}
\date{due: 3 September 2020}

\begin{document}

\maketitle

This homework assignment is due on 3 September 2020, and should be
submitted as a single PDF file to to Gradescope.

General homework expectations:
\begin{itemize}
    \item Homework should be typeset using LaTex.
    \item Answers should be in complete sentences and proofread.
    \item This homework can be submitted as a group.
\end{itemize}

\nextprob
\collab{TODO}

In this class, we will assign groups for the group project.  Now is your chance
to weigh in on how we choose them!
\begin{enumerate}
    \item Describe the problem of choosing the groups formally, including
        describing what the input and output is.  Be sure to explain any
        properties of the output that are important (e.g., the groups are all of
        size three and everyone has the same number of characters in their first
        name).
    \item Describe, in paragraph form, the algorithm you propose.
    \item Provide this algorithm in the algorithm environment.
    \item Prove that your algorithm terminates.
\end{enumerate}

\paragraph{Answer}

% ============================================

\begin{enumerate}
	\item As input, the group choosing problem consists of a set of
		students, each with unique ability levels and interests. 
		Inevitably, a wide range of ability levels will exist, and
		ability levels can be difficult to measure. (Grades and performances
		on specific assignments may be somewhat arbitrary and may not
		fully reflect a student's knowledge in a specific subject.) 
		However, a student's interest in different subject areas is 
		perhaps a more useful metric to keep in mind when determining 
		groups, since this may be determined fairly explicitly in a 
		survey kind of environment. It is my conjecture then also that group
		satisfaction in the end will be more likely optimized given interest
		in any particular subject as the main metric for assigning groups.
		Furthermore, we also have some value for group sizes as an input.
		Interestingly, half of our input (the set of students) has free
		will. A property that we can utilize to do relatively little work
		in the end in our algorithm.

		As output, we expect to have a number of groups such that each
		group is of the desired group size (with potential exceptions for
		remainders, in which case groups may have no more than 1 extra member)
		and each group is comprised of students with some level of interest
		in the topic area that group will pursue.

	\item Algorithmically, I propose using the interests of students as the primary
		force in assigning groups. A way to go about this 
		could be to open subject area proposals on a group forum.
		To start, anyone interested in a particular area could register
		this subject in the forum, or students could join in with already
		registered subject area ideas. If $S$ is the number of students in the 
		class and $k$ is the desired group size, cap each group's size at $k+1$,
		and then the next task will be to
		limit the number of total groups to $floor(S/k)$. Letting $n_0$ be the number
		of initial groups, force group members from the smallest $(n_0 - floor(S/k))$
		groups to join the largest $floor(S/k)$ groups.
	\item More formally, this algorithm can be presented as follows:

		\begin{algorithm}
			\caption{\textsc{GroupChooser}}
			\begin{algorithmic}[1]
				\Require A set $C$ of students with free will, desired group size $k$
				\Ensure A set of $k$ groups of students $G$, each of size $k$ or $k+1$
				\State $G \gets \emptyset$
				\For {student $\in C$}
					\State $i \gets 0$
					\State $j \gets |G|$
					\If{$g_0 \in G$ is interesting to the student}
						\State student $\in g_0$
					\EndIf
					\While{$g_i$ is uninteresting to the student \&\& $(i + 1) \leq k$ }
						\If {$g_{i+1}$ is interesting to the student \&\& $|g_{i+1}| < k$}
							\State student $\in g_{i+1}$
						\EndIf
						\State $i++$
					\EndWhile
					\State create $g_j \in G$
					\State student $\in g_j$
				\EndFor
				\State \textsc{Sort($G$)} (with smallest group at index 0, to largest group)
				\State $i \gets 0$
				\While {$i < (|G| - floor|C|/k)$}
					\For{student $\in g_i$}
						\State $j \gets (|G|-floor|C|/k)$
						\For{$g_j \in G$}
							\If{$g_j$ is most interesting remaining group for student}
								\State student $\in g_j$
								\State $j++$
							\EndIf
						\EndFor
						\State $i++$
					\EndFor
					\State $G \gets G/g_i$ such that $i < (|G| - floor|C|/k)$ for all integers $i \geq 0$
				\EndWhile
				\State Return $G$
			\end{algorithmic}
		\end{algorithm}

	\item \textsc{GroupChooser} terminates.
		\begin{proof}
			Let there be $n$ students in $C$.
			Lines 2 through 16 iterate through all $n$ students in the class. Embedded in these lines
			are all constant time operations, with the exception of a while loop, which may at most
			iterate $n$ times if every student creates their own group. Causing the first 16 lines to
			operate in $O(n^2)$ time.

			Line 17 sorts each group from smallest to largest, a step which occurs in $O(nlogn)$ time.

			Lastly, lines 19 through 31 comprise a  while loop, which may also at most iterate $n$ times.
			Within this loop, it is possible that every student will be iterated over in the embedded
			for loop, and then within this for loop it is posible that every group may be iterated
			over, which is also an order of $n$ operation. All other steps within this highest level
			while loop all are constant time operations. Leaving lines 19 through 31 to operate in
			$O(n^3)$ time.

			The only lines we have yet to consider are thus lines 1, 18, and 32, which all are
			constant time operations. Thus, the runtime of \textsc{GroupChooser} is
			$O(n^2) + O(nlogn) + O(n^3) = O(n^3)$. And, since we've shown a runtime, we've proven
			that \textsc{GroupChooser} terminates.
		\end{proof}
\end{enumerate}

% ============================================

\nextprob
\collab{TODO}

Chapter 1, Problem 37 (Largest Complete Subtree).

\begin{enumerate}
    \item Describe the problem in your own words, including
        describing what the input and output is..
    \item Describe, in paragraph form, the algorithm you propose.
    \item Provide this algorithm in the algorithm environment.
    \item What is the runtime of your algorithm?
    \item Prove that the algorithm is correct.
\end{enumerate}

\paragraph{Answer}

% ============================================

\begin{enumerate}
	\item Problem Description:

		If we're given a binary tree as an input with any number of nodes and binary branching structure,
		we want to give the largest complete subtree as output. More specifically, our input includes a
		set of nodes, each with links to their children, and each node also carries with it an attached
		and initially empty value for depth. As for our output, recalling the definition of the largest
		complete subtree, we want to output the largest possible connected subgraph of our
		binary tree such that every node has two children, with the exception of the leaves which all must
		have the same depth within the binary tree. This output will be in the form of the node at which
		the largest complete subtree begins, and an integer value for the depth of this largest complete
		subtree.

	\item I propose an algorithm \textsc{LargestCompleteSubtree} which will iterate through every node in 
		the given tree, and at every node will execute a recursive subalgorithm \textsc{RecursiveDepth}, 
		whose input will be the current node and then a depth value beginning at zero. If the node
		inputted to \textsc{RecursiveDepth} has no connected children, then \textsc{RecursiveDepth}
		stores the current depth value in a list provided on input. 
		Otherwise, \textsc{RecursiveDepth} adds one to the depth value, 
		and recurses on itself with each child node of the current node as input
		alongside the newly updated depth value and the most recent list of depths. 
		Eventually, the every node in this recursion will store its depth in the list, 
		and then we may calculate the depth value of the largest complete subtree
		of the current node in our original for loop in \textsc{LargestCompleteSubtree} by finding
		the minimal depth recorded in the list at a terminal node. (We know we can guarantee at least this
		depth for every other terminal node in the subtree). We assign this minimal
		depth value to the depth value associated with the current node. After every node has computed its
		largest complete subtree, we iterate through every node and locate the node with the maximum depth
		associated with its largest complete subtree, and we return this maximal node- carrying with it a
		value for the depth of its largest complete subtree.
	\item \begin{algorithm}
			\caption{\textsc{LargestCompleteSubtree}}
			\begin{algorithmic}[1]
				\Require{A binary tree $T$ containing a set of nodes with links to each of their children and an empty depth value}
				\Ensure{A node which is the start of the largest complete subtree in $T$, and with it its associated depth value}
				\State depthlist $\gets \emptyset$
				\For{each node $\in T$}
					\State \textsc{RecursiveDepth}(node, 0, depthlist)
				\EndFor
				\State max $\gets 0$
				\State MaxNode $\gets \emptyset$
				\For{each node $\in T$}
					\If{node.depth $\geq$ max}
						\State MaxNode $\gets$ node
					\EndIf
				\EndFor
				\State \Return MaxNode (with Maxnode.depth = depth of largest complete subtree)
			\end{algorithmic}		
	\end{algorithm}

	\begin{algorithm}
		\caption{\textsc{RecursiveDepth}}
		\begin{algorithmic}[1]
			\Require{node $\in T$, depth, depthlist}
			\Ensure{depth value of largest complete subtree of input node}
			\If{node.children $= \emptyset$}
				\State depthlist.add(depth)
			\EndIf
			\If{node.children $\not= \emptyset$}
				\State Child1 $\gets$ node.left
				\State Child2 $\gets$ node.right
				\State depth += 1
				\State \textsc{RecursiveDepth(Child1, depth, depthlist)}
				\State \textsc{RecursiveDepth(Child2, depth, depthlist)}
			\EndIf
			\State mindepth $\gets$ depthlist[0]
			\For {each depth $\in$ depthlist}
				\If{depth $<$ maxdepth}
					\State mindepth $\gets$ depth
				\EndIf
			\EndFor
			\State \Return mindepth
		\end{algorithmic}
	\end{algorithm}

\item \textsc{LargestCompleteSubtree} runs in $O(n^2)$ time, where $n$ is the number of nodes in the
	provided binary tree.
	\begin{proof}
		The for loop in lines 2 through 4 completes $n$ operations, each of which call 
		\textsc{RecursiveDepth}. \textsc{RecursiveDepth} at most passes through every node
		in the tree. After this is accomplished, \textsc{RecursiveDepth} iterates through
		every depth recorded in the list of depths (which are recorded at each leaf in the tree), 
		which could also be at most an order of $n$ operation. Meaning that 
		\textsc{RecursiveDepth} runs in $O(2n)$ time, and must terminate once every node called has 
		no children (which is a guarantee in a finite binary tree), if we keep moving to new child nodes.
		This makes lines 2 through 4 of \textsc{LargestCompleteSubtree} occur in $O(n*2n) = O(n^2)$ time.
		After this the only other loop in \textsc{LargestCompleteSubtree} occurs in lines 7 through 16,
		which conducts order of $n$ constant time operations. Outside of these two loops, only constant
		time operations are conducted. Making \textsc{LargestCompleteSubtree} run in $O(n^2+n)=O(n^2)$ time.
	\end{proof}
\item \textsc{LargestCompleteSubtree} Returns the largest complete subtree contained in the provided binary tree $T$
	in the form of the node where the largest complete subtree begins, and the depth of this largest complete
	subtree.
	\begin{proof}
		Begin by considering that at every node, we seek to recursively compute the largest complete
		subtree at that node. If indeed we compute the largest complete subtree at every node, then we
		should certainly be capable of returning the largest overall complete subtree in the desired
		format. Then, at every node, consider that we compute the largest complete subtree by checking
		whether or not that node has children. If the node has children, we recursively keep moving down
		the tree (starting with recursions on the leftmost child at every node), until the node we reach
		has no children. At this terminal node, we record the depth we reach in a list, which is continually
		passed through the recursion. We can be sure that the list records the depth at every terminal node
		branching from our original node because there is order in the way recording occurs, as the list
		is updated moving from left to right and values must be carried from left to right in the recursion.
		The earliest changes in the list must be traceable to the leftmost terminal node, and any change 
		to the list thereafter happens in a routine called after the initial change to the list 
		(when the right child is called), so our data is preserved in the list for the depth of every
		terminal node. Then, we can simply choose the shallowest terminal node (because we can guarantee
		that there is a node on this level throughout the rest of the subtree) to attain our largest
		complete subtree for any given original node. We return this in the form of a depth integer
		value which is then stored alongside the original node. Afterwards, simply iterating through
		every node and determining which node has the largest accompanying depth value of its subtree
		provides the overall largest complete subtree for the entire binary tree. As a result, we can
		guarantee that \textsc{LargestCompleteSubtree} does indeed provide the largest complete subtree
		in the specified output format upon a desired input, since we can verify that we correctly compute
		the largest complete subtree for any given node in the graph, and then return the largest complete
		subtree when examining every node.
	\end{proof}


\end{enumerate}
% ============================================


\nextprob
\collab{TODO}

Chapter 1, Problem 9 (Pancakes). When describing your algorithm, please give a
prose explanation (in paragraph form) as well as in the algorithm environment.
To ``Pass", we expect and answer to (a) and (b).  To earn a ``high pass" on this
question, you must answer (c) as well.

<<<<<<< HEAD
\paragraph{answer}
\begin{enumerate}[a]
	\item
	Let our stack of pancakes have size rankings 1 through $n$, randomly distributed
	throughout the stack. If we only have $n=1$ pancake, we know we don't have to do
	anything, our stack is already sorted. If we have $n=2$ pancakes, we can sort the
	stack in only one flip, simply by flipping the stack so that the largest pancake is
	on the bottom if it isn't already. If we have $n \geq 3$ pancakes, then we will locate
	the largest pancake in the stack, flip every pancake ranging from the top of the stack to the
	largest pancake so that the largest pancake is on top, and then flip the entire stack so that the
	largest pancake is on the bottom of the stack. Then, we will recursively recall the algorithm
	with $n-1$ pancakes, no longer considering the bottom-most sorted region of the stack. After all
	of our recursions are complete, we should have a fully sorted stack of pancakes with largest on the
	bottom, smallest on the top.


	\begin{algorithm}
		\caption{\textsc{PancakeSort}}
		\begin{algorithmic}[1]
			\Require A list $l_n$ of $n$ pancakes ranging from 1 to $n$ in their index in the list, in any order based on size
			\Ensure A sorted list of $n$ pancakes with largest on the bottom, smallest on top
			\If{$n=1$}
				\State \Return $l_n$
			\EndIf
			\If{$n=2$}
				\If{pancakes sorted}
					\State \Return $l_n$
				\Else
					\State flip $l_n$
					\State \Return $l_n$
				\EndIf

			\EndIf
			\If{$n\geq 3$}
				\State find largest pancake at index $k$
				\State flip 1st through $kth$ elements
				\State flip entire stack
				\State $l_{n-1} \gets (l_n - nth$ element) (list discluding largest pancake, now at the bottom of the stack)
				\State \textsc{PancakeSort}($l_{n-1}$)
			\EndIf
		\end{algorithmic}
	\end{algorithm}


	Letting $T(n)$ be the number of flips our algorithm uses to sort a stack of $n$ pancakes, we know that
		$T(n) = 0$ if $n=1$, $T(n)\leq1$ if $n=2$, and $T(n)\leq 2+ T(n-1)$ for $n \geq 3$. So I guess if
		we wanted to think about exactly how many flips would occur in the worst case, it'd be roughly 
		$2n$, if somehow we both needed two flips in every case to get the largest current pancake to the 
		bottom of the stack, and then one additional flip when we only had two pancakes remaining.
\item For every positive integer $n$, we can think of a stack that needs at least $n$ flips to sort by considering
	a stack with the largest pancake directly in the middle of the stack, and then the smallest two pancakes on 
		either side of it, and then the next smallest pancake above the smallest pancake on each side, and
		then so on until the next largest pancakes apart from the pancake in the middle are the furthest
		outside. For example, a stack of 5 pancakes ranging from size 1 to 5 would have the configuration
		\{3,1,5,2,4\}, where 3 is on top and 4 is on the bottom to begin. Then, we verifiably need a 
		porportional to linear number of flips for any $n$, because in having the largest value in the
		middle, and next largest values on the extremities of the stack, we are forcing distance between
		the current largest value at any moment and the next largest value. In doing so, we are causing
		a large number of flips, which if not occuring for every value twice must at least occur for every
		value in the stack once. Since a flip must occur in general for a value at least once, we know we
		have something that is porportional to $n$ flips in the best case.
	\item For this extension of the problem, we can do something quite similar to the technique used before.
		This time, whenever we flip the largest pancake to the top of the stack, we can also just verify
		that the burned side is facing up, so that upon flipping the entire stack, the pancake's burned side
		will be facing down.
		When the base cases are reached, if there is only one remaining pancake, we simply flip it if the
		burned side is up. For two remaining pancakes, we first make sure the stack is sorted, flipping it if it isn't,
		and then if both pancakes are burnt we flip the stack again, flipping the largest burned pancake, flip the stack
		so that it's in order again, and then flip the smallest burnt pancake on top. If the bottom pancake is the only
		one burnt, we flip the stack, flip the pancake, and flip the stack again. If the top pancake is the only one burnt
		then we just flip the top pancake. Otherwise, if no pancakes are burnt, we return the stack as is.

		
	\begin{algorithm}
		\caption{\textsc{BurnedPancakeSort}}
		\begin{algorithmic}[1]
			\Require A list $l_n$ of $n$ pancakes ranging from 1 to $n$ in their index in the list, in any order based on size, with one side of each pancake burned
			\Ensure A sorted list of $n$ pancakes with largest on the bottom, smallest on top, and burned sides of every pancake facing down
			\If{$n=1$}
				\If{burned side is up}
					\State flip the one pancake in $l_n$
					\State \Return $l_n$
				\Else
					\State \Return $l_n$ as is
				\EndIf
			\EndIf
			\If{$n=2$}
				\If{pancakes sorted}
					\If{top and bottom pancakes both have burn side up}
                                        \State flip stack
					\State flip top pancake
                                        \State flip stack
                                        \State flip top pancake
                                        \State \Return $l_n$
					\EndIf

					\If{just bottom pancake has burn side up}
					\State flip stack
					\State flip top pancake
					\State flip stack
					\State \Return $l_n$
					\EndIf
					\If{just top pancake has burn side up}
					\State flip top pancake
					\State \Return $l_n$
					\EndIf
					\State \Return $l_n$
				\Else
					\State flip $l_n$
					\If{top and bottom pancakes both have burn side up}
                                        \State flip stack
                                        \State flip top pancake
                                        \State flip stack
                                        \State flip top pancake
                                        \State \Return $l_n$
                                        \EndIf
					\If{just bottom pancake has burn side up}
                                        \State flip stack
                                        \State flip top pancake
                                        \State flip stack
                                        \State \Return $l_n$
                                        \EndIf
                                        \If{just top pancake has burn side up}
                                        \State flip top pancake
                                        \State \Return $l_n$
                                        \EndIf
                                        \State \Return $l_n$

				\EndIf

			\EndIf
		\end{algorithmic}
	\end{algorithm}
	\begin{algorithm}
		\caption{Part 2 of the algorithm (beginning at line 51)}
		\begin{algorithmic}[1]
			\If{$n\geq 3$}
				\State find largest pancake at index $k$
				\State flip 1st through $kth$ elements
				\If{top element has burned side down}
					\State flip top pancake
				\EndIf
				\State flip entire stack
				\State $l_{n-1} \gets (l_n - nth$ element) (list discluding largest pancake, now at the bottom of the stack)
				\State \textsc{PancakeSort}($l_{n-1}$)
			\EndIf
		\end{algorithmic}
	\end{algorithm}

	As for how exactly this performs in the worst case, if $T(n)$ is the number of flips we'll need to do,
	then $T(n) \leq 1$ if $n=1$, $T(n) \leq 5 + T(n-1)$ if $n=2$, and then $T(n) \leq 3 + T(n-1)$ for all $n$
	greater than 2. So, in the worst case, we have roughly $3n$ flips.
		
\end{enumerate}		
=======
\paragraph{Answer}

% ============================================

TODO: your answer goes between these lines

% ============================================


>>>>>>> 6ef736806df025e7a80b449def37dee559e01d9a

\end{document}

