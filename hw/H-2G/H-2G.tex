\documentclass{article}
\usepackage{../fasy-hw}
\usepackage{ wasysym }

%% UPDATE these variables:
\renewcommand{\hwnum}{2}
\title{Advanced Algorithms, Homework 1}
\author{TODO-Group Member Names Here}
\collab{n/a}
\date{due: 3 September 2020}

\begin{document}

\maketitle

This homework assignment is due on 3 September 2020, and should be
submitted as a single PDF file to D2L and to Gradescope.

General homework expectations:
\begin{itemize}
    \item Homework should be typeset using LaTex.
    \item Answers should be in complete sentences and proofread.
    \item This homework can be submitted as a group.
\end{itemize}

\nextprob
\collab{TODO}

In this class, we will assign groups for the group project.  Now is your chance
to weigh in on how we choose them!
\begin{enumerate}
    \item Describe the problem of choosing the groups formally, including
        describing what the input and output is.  Be sure to explain any
        properties of the output that are important (e.g., the groups are all of
        size three and everyone has the same number of characters in their first
        name).
    \item Describe, in paragraph form, the algorithm you propose.
    \item Provide this algorithm in the algorithm environment.
    \item Prove that your algorithm terminates.
\end{enumerate}

\paragraph{Answer}

% ============================================

TODO: your answer goes between these lines

% ============================================

\nextprob
\collab{TODO}

Chapter 1, Problem 37 (Largest Complete Subtree).

\begin{enumerate}
    \item Describe the problem in your own words, including
        describing what the input and output is..
    \item Describe, in paragraph form, the algorithm you propose.
    \item Provide this algorithm in the algorithm environment.
    \item What is the runtime of your algorithm?
    \item Prove that the algorithm is correct.
\end{enumerate}

\paragraph{Answer}

% ============================================

TODO: your answer goes between these lines

% ============================================


\nextprob
\collab{TODO}

Chapter 1, Problem 9 (Pancakes). When describing your algorithm, please give a
prose explanation (in paragraph form) as well as in the algorithm environment.
To "Pass", we expect and answer to (a) and (b).  To earn a "high pass" on this
question, you must answer (c) as well.

\end{document}

