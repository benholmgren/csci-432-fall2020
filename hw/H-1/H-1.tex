\documentclass{article}
\usepackage{../fasy-hw}
\usepackage{ wasysym }

%% UPDATE these variables:
\renewcommand{\hwnum}{1}
\title{Advanced Algorithms, Homework 1}
\author{Ben Holmgren}
\collab{n/a}
\date{due: 27 August 2020}

\begin{document}

\maketitle

This homework assignment is due on 27 August 2020, and should be
submitted as a single PDF file to D2L and to Gradescope.

General homework expectations:
\begin{itemize}
    \item Homework should be typeset using LaTex.
    \item Answers should be in complete sentences and proofread.
\end{itemize}

\nextprob
\collab{n/a}

Answer the following questions:
\begin{enumerate}
    \item What is your elevator pitch?  Describe yourself in 1-2
                sentences.
    \item What was your favorite CS class so far, and why?
    \item What was your least favorite CS class so far, and why?
    \item Why are you interested in taking this course?
    \item What is your biggest academic or research goal for this semester (can
        be related to this course or not)?
    \item What do you want to do after you graduate?
	\item What was the most challenging aspect of your coursework last semester
        after the university transitioned to online?
    \item What went well last semester for you after the university transitioned
        to online?
\end{enumerate}

\paragraph{Answer}

% ============================================
	\begin{enumerate}
		\item Growing up in Bozeman, being in the outdoors has always been an important
			part of my life, and in particular I love to spend my time trail running
			and climbing rock and ice. I am also fascinated by the intersection of pure
			mathematics and computation, and when not outdoors spend the majority of my
			time on problems in this realm. There is nothing better in my opinion than
			watching oneself improve at a desired task, and I certainly hope that my life
			will always be a story of growth- especially in these two important facets.

	    	\item Probably computational topology, of course. Either that or CS theory.
		I really like a lot of theory, and both of these classes provided that.
		I love to understand the bounds of computation, to design algorithms, 
		and to attempt to conceptualize abstract things. I liked graphics a lot too
		because I felt like I developed a lot as a programmer, but there was too much
		implementation for it to have been explicitly a favorite above the other two.
	    	
		\item I really didn't like the multidisciplinary engineering design course
		that we need to take- if that even counts as a CS class. I ended up
		having to do a lot of wiring and its just not work that I feel like I
		thrive at or enjoy. That class had nothing that I felt was difficult
		to understand or implement, but was still very time consuming, which
		was a recipe for some frustration.
	    	
		\item Because designing algorithms is a ton of fun, and it's something that I
		could definitely enjoy doing as a career. It turns out that there are people who
		work for big tech companies who get paid to do that all day! It also turns out that
		there are people who get paid to do a whole lot of that at a huge number of universities!
		Sounds like the dream!
	
		\item I'd really like to get another publication, and as of tonight Brad and I
		might have another good result to write about that comes from an open problem proposed
		in CCCG by Joe O'Rourke! In particular, I'd really like
		to recieve a publication that is entirely student written, probably with the
		one and only Bradley McCoy. I'm also going to try my best this fall to get a 
		goldwater scholarship, which is another major objective. The list goes on and on
		with this one, so I will refer you to past and future conversations between us for 
		the rest of this answer.
	        
		\item I want to pursue a PhD, hopefully at the University of Washington. This seems like
		a very strong program to me, and the best location for me to attend graduate school
		at this stage in my life and looking towards the future, where I'd hope to have some
		connections and be able to remain in the Pacific Northwest for a while.

		\item The most challenging part of last semester after the transition was definitely missing
		all of the close friends that I get to learn with and do research with every day, especially
		with how extremely difficult the semester was for our group. Those connections mean a lot,
		and especially with the situation that we began all of this in, that was severely missed.
		But I'm more used to the distance now, and have much better figured out how to navigate
		this for myself.
		
		\item I managed my time pretty well. I often work by getting really excited about a subject
		and taking a deep dive in it for an entire day or so. The online transition actually
		allowed for this much better than in person classes, where you are required to be in
		class in person and feel for some reason more obligated to work more incrementally on
		different things, rather than just going huge in one subject area when you feel like it.

	\end{enumerate}
% ============================================

\nextprob
\collab{n/a}

Please do the following:
\begin{enumerate}
    \item Write this homework in LaTex.
        Note: if you have not used LaTex before and this is an
        issue for you, please contact the instructor or TA.
    \item Update your photo on D2L to be a recognizable headshot of you.
    \item Sign up for the class discussion board.
\end{enumerate}

\paragraph{Answer}

% ============================================

\begin{enumerate}
	\item Done.
	\item Done.
	\item Done.
\end{enumerate}

% ============================================


\nextprob
\collab{}

    In this class,
    please properly cite all resources that you use.
    To refresh your memory on what plagiarism is,
    please
    complete the plagiarism tutorial found here:
    \url{http://www.lib.usm.edu/plagiarism_tutorial}.
    If you have observed plagiarism or cheating in a classroom (either as an
    instructor or as a student), explain the situation and how it made you
    feel.  If you have not experienced plagiarism or cheating or if you would
    prefer not to reflect on a personal experience, find a news
    article about plagiarism or cheating and explain how you would feel if you
    were one of the people involved.


\paragraph{Answer}

% ============================================

When I was a junior in highschool, I took was in an AP US history class. I ended up 
getting a perfect score on the first exam, and I had scored much higher than most people
in the class. I was sitting next to a few people who were clearly struggling in the class,
and were also the pretty obvious ``popular" kids in highschool. Pretty quickly I was getting 
asked always for answers to tests and quizzes and homework, and had to deal often with the
feeling of wandering eyes on my paper. More than anything, it was just a major pain. This carried
on for a bit, but I tried to just ignore it and eventually moved seats. I was a little annoyed that
people weren't earning their test answers, but more than anything was just underwhelmed and disappointed
with the laziness of people. But life goes on, and life went on. I think that academic dishonesty
is just generally pretty stupid because you don't get anything out of the class- which is the entire
point of being there. So I guess that's my big takeaway, and luckily it hasn't really been a part of my life
since that point in time.

% ============================================



\nextprob
Prove the following statement: Every tree with one or more nodes/vertices has
exactly $n-1$ edges.

\paragraph{Answer}

% ============================================
\begin{theorem} Every tree with one or more nodes/vertices has
exactly $n-1$ edges.
\end{theorem}
\begin{proof}
	Let $T$ be a tree with $n=1$ vertices as a base case. Then, $T$ most definitely has exactly $n-1=0$ 
	edges, since a tree by definition is an acyclic graph in which any two vertices are connected with exactly one path.
	Since $T$ has only one vertex, $T$ then must have zero edges to fit the definition of a tree, since all 
	vertices in $T$ must be connected (which occurs trivially), and $T$ must be acyclic, so no self loops may occur.

	Now let $T$ be a tree with $n=k$ vertices for some $k>1$. Assume that that the theorem holds for a tree with $k-1$ vertices.
	We need to show that the theorem holds for a tree on $k$ vertices.

	Choose any degree 1 vertex $v$. Let a graph $T'$ be the graph resulting from the removal of $v$ and its single connected
	edge, $e$. $T'$ must also be an acyclic, fully connected graph, since all we've done is remove one vertex and its single
	corresponding edge on the very periphery of $T$. Meaning that $T'$ must also be a tree with $k-1$ vertices. Then as a 
	result, $T'$ has $k-2$ edges by the inductive hypothesis. Which means that $T$ is a tree with $(k-2) + 1 = k - 1$ edges. 
	
	Because the base case and inductive hypothesis hold, the theorem is proven by induction.


\end{proof}

% ============================================



\nextprob
Use the definition of big-O notation to prove that $f(x)=n^2 + 3n +2$ is
$O(n^2)$.

\paragraph{Answer}

% ============================================

\begin{theorem}
	$f(x)=n^2 + 3n +2$ is $O(n^2)$.
\end{theorem}
\begin{proof}
	Recall that by definition, $f(x) = O(g(x))$ if there exists a positive real number $K$ and a real number $x_0$
	such that

	$|f(x)|\leq K g(x)$ for all $x \geq x_0$.

	Let $f(x)=n^2 + 3n + 2$ and $g(x) = n^2$.

	Consider $K=6$ and $x_0=1$. For $x=x_0$, $f(x)=6=K(g(x))$. For all $x > x_0$ then, $f(x) \leq K g(x)$, which may
	be shown graphically. 

	Therefore, by the definition of big-O notation, $f(x)=n^2 + 3n + 2$ is $O(n^2)$.

\end{proof}

% ============================================



\nextprob
Consider the \textsc{RightAngle} algorithm on page 8 of the textbook.
\begin{enumerate}
    \item When we design an algorithm, we design the algorithm to solve a
        problem or answer a question.  What is the problem that this algorithm
        solves?
    \item Prove that the while loop terminates.
\end{enumerate}

\paragraph{Answer}

% ============================================

\begin{enumerate}
	\item
\end{enumerate}

% ============================================



\nextprob
Consider the following statement: If $a$ and $b$ are both even numbers, then $ab$ is
an even number.
\begin{enumerate}
    \item What is the definition of an odd number?
    \item What is the definition of an even number?
    \item What is the contrapositive of this statement?
    \item What is the converse of this statement?
    \item Prove this statement.
\end{enumerate}

\paragraph{Answer}

% ============================================

TODO: your answer goes between these lines.  Be sure to enumerate!

% ============================================



\end{document}

