\documentclass{article}
\usepackage{../fasy-hw}
\usepackage{ wasysym }

%% UPDATE these variables:
\renewcommand{\hwnum}{4}
\title{Advanced Algorithms, Homework \hwnum}
\author{TODO-Put Your Name Here}
\collab{n/a}
\date{due: 6 October 2020}

\begin{document}

\maketitle

This homework assignment should be
submitted as a single PDF file to to Gradescope.

General homework expectations:
\begin{itemize}
    \item Homework should be typeset using LaTex.
    \item Answers should be in complete sentences and proofread.
    \item This homework can be submitted as a group.
\end{itemize}

\nextprob
\collab{TODO}

You should make at least ten contributions to the Piazza board
discussing the solutions to Problems in Chapter 3 of the textbook.  Your
contribution does not have to be a complete solution.  It can be any element of
a full solution to a problem requiring an algorithm as an answer.  (For this
question, the outcomes are: insufficient posts (-1), low pass (+1), pass (+3),
and high pass (+5).

As a reminder, a full solution to a textbook problem will have the following elements:
\begin{enumerate}
    \item Describe the problem in your own words, including
        describing what the input and output is.
    \item Describe, in paragraph form, the algorithm you propose.
    \item Provide a nicely formatted algorithm to solve the problem.
    \item Use a decrementing function to prove that algorithm terminates.
    \item Give the runtime with justification.
    \item If there is a loop or recursion, what is the loop/recursion invariant? Provide the proof.
\end{enumerate}

\paragraph{Answer}

% ============================================

My contributions are:
\begin{enumerate}
    \item (TODO: state the problem number, and date/time). TODO:
        copy the post here.
    \item ...
\end{enumerate}

% ============================================

\nextprob
\collab{TODO}

Choose one of the Chapter 3 problems discussed in Piazza and provide a solution
in your own words.  This should be a polished solution.

\paragraph{Answer}
% ============================================

TODO: your answer goes between these lines

% ============================================

\nextprob
\collab{TODO}

For this problem, choose either the Edit distance algorithm (Section 3.7) or the
Subset Sum problem (Section 3.8). Look at three different sources (including the
textbook) that describe
and analyze the same algorithm. In one to two pages, describe the similarities
and differences in the presentation and analysis of the algorithms.

\paragraph{Answer}

% ============================================

TODO: your answer goes between these lines

% ============================================

\end{document}
